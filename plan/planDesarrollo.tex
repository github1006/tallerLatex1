
%\ aquí se menciona el proposito, descripción del producto
\addcontentsline{toc}{chapter}{Introducción}
\chapter*{Introducción}
\stepcounter{chapter}
Una vez que es completada los requerimientos del proyecto y los objetivos que es el desarrollo del Sistema Software de Gestión de Farmacia. Con tiempo y presupuesto limitados. En aras de la satisfacción de la cliente se reunió con todos los miembros del proyecto para estructurar los puntos principales y asignación de roles. 

\section{Descripción del proyecto}
Actualmente el estudiante posee los libros digitales o digitalizados y otros como videos, artículos en un disco duro de 2 TB organizado en carpetas sin acceso fácil a ellos. En la actualidad sólo es
posible hacer un link con un programa llamano Kiwix Desktop con el cual es posible acceder con un click al libro en formato pdf o cualquier otro video o artículo o audio (aquí llamados documentos). Este método es inefectivo ya que requiere mucho tiempo para hacer los links. Además cuando se cambia el nombre de la carpeta superior o del mismo se pierde el acceso. 
Los libros están en diferentes formatos como ser: pdf, pub, djvu, pero mayormente en pdf, algunos en html. 
Actualmente se lo tiene en discos duros y en memoria de dispositivos móviles como tablets, celular o utro dispositivo, muchos ocupan mucha memoria porque hay redundacia de datos ya que una copia está en un disco y otra copia del mismo en otro dispositivo y alguno se corrompe al hacer copias y cerrar sin guardar un documento que se estaba viendo .

Las videos, fueron descargados para que tengan buena calidad y sean de latencia mínima al reproducirlos, aunque no se tiene gran cantidad, pero en el futuro esto se incrementará. Los videos se tienen en discos rígidos externos; ésto para no sobrecargar el almacenamiento de la computadora personal.
En cuanto a imágenes no se tiene mucho, sin embargo como el alumno suele ser más visual en el método de aprendizaje también se provee tener una base de datos de imágenes importantes y relevantes a cada tema.

En otros, se tiene herramientas como calculadoras, IDEs, CASE, diagramadores y páginas web relacionados con simulación y otros necesarios para aprendizaje del estudiante, están desorganizados y el alumno pierde buscando lo que una vez ya había encontrado en la web. 
\section{Necesidades del proyecto}
El alumno o estudiante tiene mucha información generada durante su paso como estudiante pero éstas están desorganizadas y el acceso a veces se vuelve de forma que no se puede encontrar lo que se busca. También si el alumno tiene ordenado algo en su computadora, cuando se encuentra lejos de donde estudia, no puede acceder y debe llevar copias pero a veces copia un contenido que tarda en copiar,  o simplemente copió otro contenido y un sin fin de problemas que puede tener el no accesos oportuno a la información. 
%\section{Abreviaturas}
\section{Propósito u objetivo}
El objetivo es realizar el desarrollo de una herramienta de estudio efectiva para población de diferentes niveles para ayudar en el  proceso de aprendizaje.  
Y los objetivos específicos son:
\begin{itemize}
	\item Hacer una biblioteca digital más flexible y adaptable posible.
	\item Seleccionar organizar y mantener objetos digitales para el estudiante
\end{itemize}

\section{Alcance}
En primera instancia sólo se desarrollará para documentos tales como artículos y libros, sin embargo para versiones posteriores será extensible a cualquier objeto digital que el alumno requiera para su aprendizaje. 
\addcontentsline{toc}{chapter}{Vista general del proyecto}
\chapter*{Vista general del proyecto}
\stepcounter{chapter}
%El proyecto de FARMAKUM incluye el desarrollo de software AdmiFarm, un sistema de inventario y facturación alojado en la nube. Admifarm provee los requisitos de comunicación segura
El proyecto de biblioteca digital incluye el desarrollo de software Biblioteca2295 con un sistema de inventario alojado en la nube.
\section{Suposiciones y restricciones}
%Pasos para control de proyecto
\addcontentsline{toc}{chapter}{Organización del Proyecto}
\chapter*{Organización del Proyecto}%19.2.1 Project Organization
El proyecto se organiza con personas que tienen algún conocimientos a cerca de bibliotecas digitales, se busca también algunos con habilidades específicas. 

\stepcounter{chapter}

\section{Participantes en el proyecto}
%https://monday1006.monday.com/boards/3478543214  aquí se puede hacer una tabla
\subsection{Jefe de proyecto} Es el que se encargará del coordinar con todos y ser puente entre el cliente y el equipo.
\subsection{Analista de sistemas} 
Se encarga del modelado funcional, estructural y de comportamiento.
Realiza el modelado de procesos, modelos físico y lógicos (diagrama de casos de uso) 
\subsection{Diseñador  de sistemas} 
Se ocupa de validación, análisis de modelos, estrategias de diseño, como ser el diseño orientado a objetos, capas, interacción computadora-humano y de capa de arquitectura física. 
\subsection{Desarrolladores}
Los desarrolladores se harán cargo de las tecnologías de implementación, crear clases y objetos. También se hacen cargo de manejo de datos y verificación de software. 
\subsection{Tester}
Se encargarán de la calidad de software, proceso de pruebas, pruebas dinámicas y estáticas e implementar herramientas de testeo.
\section{Roles y responsabilidades}
%Tabla con posicion y rol o responsabilidad
\begin{table}
\begin{tabular}{ll} \vline
Posición & Responsabilidad o Rol \\ \vline
\end{tabular}
\end{table}

\addcontentsline{toc}{chapter}{Gestión de proceso}
\chapter*{Gestión de proceso}
\stepcounter{chapter}
%en ingles gestion es management
\section{Estimación del proyecto}
%Faces de desarrollo
%Objetivos
%Plan de despliegue o liberación
%\subsection{Estimación y gestión de recursos humanos}
%\subsection{Estimación de recursos de software}
%\subsection{Estimación del tiempo}
\section{Herramientas para gestión y ciclo de vida de software de proyectos}
\subsection{Herramientas de control de código fuente}
Se utilizará el git como software de control de versiones. 

Se utilizará GitHub como alojamiento de código y de repositorio.
\section{Plan de proyecto}
\subsubsection{Calendario de proyecto}
\subsubsection{Diagrama de Gantt}
%https://www.teamgantt.com/

\section{Seguimiento y control de proyecto}
%\chapter*{Documentación}


