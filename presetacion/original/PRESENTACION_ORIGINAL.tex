\documentclass[article]{beamer}
\usetheme{Warsaw}
\setbeamertemplate{footline}[frame number]


\usefonttheme[]{serif}
\usepackage{amsmath, latexsym, color, graphicx, amssymb, bm, here}
\usepackage{epsf, epsfig, pifont,tikz,subfigure}
\usepackage{graphics, calrsfs}
\usepackage{times}
\usepackage{fancybox,calc}
\usepackage{palatino,mathpazo}
\usepackage{amsfonts}
%\usepackage{wrapfig}
%\usepackage{multicol}
\usepackage{sidecap}



\title{Presentation using \LaTeX{} Beamer}
\author{Tao Ma}
\institute{Electrical and Computer Engineering \\ Auburn University}
\date{\scriptsize{\today}}



\AtBeginSection[]
{
  \begin{frame}{Outline}
    \tableofcontents[currentsection]
  \end{frame}
}


\begin{document}

%%%%%%%%%%%%%%%%%%%%%%%%%%%%%%Frame 1

\maketitle


%%%%%%%%%%%%%%%%%%%%%%%%%%%%%%Frame 2
\begin{frame}
\frametitle{Outline}
\tableofcontents
\end{frame}
\section{Introduction}


%%%%%%%%%%%%%%%%%%%%%%%%%%%%%%Frame 3
\begin{frame}[fragile]
\frametitle{About Beamer}
\begin{itemize}
\item  Beamer - a LaTeX class for creating presentations. 
\item  Different from WYSWYG programs. 
\item  A Beamer presentation is like any other LaTeX document:
\begin{itemize}
\item It has a preamble and a body.
\item The body contains sections and subsections. 
\item The different slides are put in environments.
\item Slides are structured using itemize and enumerate environments, or plain text.
\end{itemize} 
\end{itemize}
\end{frame}

%%%%%%%%%%%%%%%%%%%%%%%%%%%%%%Frame 4
\section{Document Structure}
\subsection{Example}


%%%%%%%%%%%%%%%%%%%%%%%%%%%%%%Frame 5

\begin{frame}[fragile]
\frametitle{A Simple Example}
\scriptsize{\tt{
\alert<2>{\textbackslash documentclass\{beamer\}\\ \vspace{0.1cm}

\textbackslash title\{\textbackslash LaTeX\{\} Beamer Class Introduction\} \\
\textbackslash author\{Tao Ma\}\\
\textbackslash institute\{Electrical and Computer Engineering \textbackslash\textbackslash Auburn University\}\\
\textbackslash date\{\textbackslash scriptsize\{\textbackslash today\}\}
} \\ \vspace{0.3cm}

\alert<3>{\textbackslash begin\{document\} \\ \vspace{0.1cm}

\textbackslash maketitle\\ \vspace{0.1cm}

\textbackslash section\{Section 1\}\\
\textbackslash begin\{frame\}\\
\textbackslash frametitle\{Frame 1 Name\}\\
\textbackslash Here is one slide.\\
\textbackslash end\{frame\}\\ \vspace{0.1cm}

\textbackslash section\{Section 2\}\\
\textbackslash frame\{\\
\textbackslash frametitle\{Frame 2 Name\}\\
\textbackslash Here is another slide.\\
\}\\ \vspace{0.1cm}

\textbackslash end\{document\}\\
}
}}
\end{frame}


%%%%%%%%%%%%%%%%%%%%%%%%%%%%%%Frame 6

\subsection{Preamble}


%%%%%%%%%%%%%%%%%%%%%%%%%%%%%%Frame 7
\begin{frame}[fragile]
\frametitle{Preamble}
\scriptsize{\tt{
\alert<2>{
\textbackslash documentclass[10pt]\{beamer\}\\
\textbackslash usetheme\{Warsaw\}\\
\textbackslash setbeamertemplate\{footline\}[frame number]\\
}
\vspace{0.5cm}
\alert<3>{
\textbackslash usefonttheme[]\{serif\}\\
\textbackslash usepackage\{amsmath, latexsym, color, graphicx\}\\
\textbackslash usepackage\{epsf, epsfig,subfigure\}\\
\textbackslash usepackage\{amsfonts,multicol\}\\
}
\vspace{0.5cm}
\alert<4>{
\textbackslash newcommand\{\textbackslash tbs\}\{\textbackslash textbackslash\}\\
\textbackslash definecolor\{cRed\}\{rgb\}\{1, 0, 0\}\\
}
}}
\end{frame}



%%%%%%%%%%%%%%%%%%%%%%%%%%%%%%Frame 8


\begin{frame}[fragile]
\frametitle{Preamble}
\framesubtitle{Choosing a Theme}

Different themes can be selected by changing the\\
\begin{verbatim}
\usetheme{Warsaw}
\end{verbatim}
\begin{block}{Themes}
{
\begin{center}
\begin{tabular}{cccc}
Antibes & Bergen & Berlin & Madrid\\ 
Paloalto 	& Pittsburgh & Rochester & Singapore \\
\end{tabular}
\end{center}
} \vspace{-0.5cm}
\end{block}
%\begin{verbatim}
%\usecolortheme{...}
%\end{verbatim}
\end{frame}

%%%%%%%%%%%%%%%%%%%%%%%%%%%%%%Frame 9
\subsection{Body}

\begin{frame}
\frametitle{Body}

\scriptsize{\tt{
\alert{\textbackslash begin\{document\}} \\ \vspace{0.2cm}
\textbackslash maketitle\\ \vspace{0.2cm}

\textbackslash section\{Section 1\}\\
\textbackslash begin\{frame\}\\
\textbackslash frametitle\{Frame 1 Name\}\\
\textbackslash Here is one slide.\\
\textbackslash end\{frame\}\\ \vspace{0.2cm}

\textbackslash section\{Section 2\}\\
\textbackslash frame\{\\
\textbackslash frametitle\{Frame 2 Name\}\\
\textbackslash Here is another slide.\\
\}\\ \vspace{0.2cm}

\alert{\textbackslash end\{document\}}\\
}}
\end{frame}



%%%%%%%%%%%%%%%%%%%%%%%%%%%%%%Frame 10

\begin{frame}
\frametitle{Making the Title Frame}

Use the following commands in your preamble:
\begin{block}{Commands for the title frame}
\scriptsize{\tt{
\textbackslash title\{...\} \\
\textbackslash subtitle\{...\} \\
\textbackslash author\{...\}\\
\textbackslash institute\{...\}\\
\textbackslash date\{...\} \\
}}
\end{block}
and the command \textbackslash \tt{maketitle}, 
\begin{block}{Generating the title frame in the body}
\scriptsize{\tt{
\textbackslash begin\{document\} \\ 
\textbackslash maketitle\\ 
\vdots
\textbackslash end\{document\}\\
}}
\end{block}
\end{frame}


%%%%%%%%%%%%%%%%%%%%%%%%%%%%%%Frame 11

\begin{frame}
\frametitle{Creating a Frame}

Creating a frame use with commands:
\begin{block}{}
\scriptsize{\tt{
\textbackslash begin\{frame\} \\
\textbackslash frametitle\{...\} \\
\textbackslash framesubtitle\{...\} \\
...\\
\textbackslash end\{frame\}
}}
\end{block}
Or, 
\begin{block}{}
\scriptsize{\tt{
\textbackslash frame\{ \\
\textbackslash frametitle\{...\}\\
\textbackslash framesubtitle\{...\}\\
...\\
\}
}}
\end{block}
\end{frame}


%%%%%%%%%%%%%%%%%%%%%%%%%%%%%%Frame 12

\begin{frame}[fragile]
\frametitle{Section and Subsections}
Section specifications are declared between the frames.\\
\vspace{1cm}
\scriptsize{\tt{
...\\
\textbackslash end\{frame\}\\
\vspace{0.5cm}
\alert{
\textbackslash section\{section name\}\\
\textbackslash subsection\{subsection name\}\\
\textbackslash subsubsection\{subsubsection name\}}\\
\vspace{0.5cm}
\textbackslash begin\{frame\}\\
...\\
}
}

\end{frame}


%%%%%%%%%%%%%%%%%%%%%%%%%%%%%%Frame 13
\begin{frame}[fragile]
\frametitle{Generating the Outline Frame}

The outline of your presentation can be added as follows,\\

\begin{block}{manually}
\scriptsize{\tt{
...\\
\textbackslash end\{frame\}\\
\vspace{0.1cm}
\textbackslash begin\{frame\}\\
\textbackslash frametitle\{Outline\}\\
\textbackslash tableofcontents[currentsection]\\
\textbackslash end\{frame\}\\
\vspace{0.2cm}
\textbackslash begin\{frame\}\\
...\\
}}
\end{block}

\begin{block}{automatically}
\scriptsize{\tt{
...\\
\textbackslash AtBeginSection[]\{\\
\textbackslash begin\{frame\}\{Outline\}\\
\textbackslash tableofcontents[currentsection]\\
\textbackslash end\{frame\}\}\\
...\\
}}
\end{block}
\end{frame}


%%%%%%%%%%%%%%%%%%%%%%%%%%%%%%Frame 14
\section{Frame Structure}
\subsection{Column}


%%%%%%%%%%%%%%%%%%%%%%%%%%%%%%Frame 15
\begin{frame}[fragile]
\frametitle{Column}
\begin{block}{Example: two columns}\scriptsize{
\textbackslash begin\{columns\}\\
\textbackslash column\{.4\textbackslash textwidth\}\\
Left column\\
\textbackslash column\{.4\textbackslash textwidth\}\\
Right column\\
\textbackslash end\{columns\}\\
}
\end{block}
\vspace{1cm}
\onslide<2>{
\begin{columns}
\column{.4\textwidth}
Left column\\
\column{.4\textwidth}
Right column\\
\end{columns}
}
\end{frame}


%%%%%%%%%%%%%%%%%%%%%%%%%%%%%%Frame 16
\subsection{Block}
\begin{frame}
\frametitle{Block}

\alert{\textbackslash begin\{block\}}\{Beamer Introduction\}\\
Beamer is a \{\textbackslash LaTeX\} class.\\
\alert{\textbackslash end\{block\}}\\
\vspace{1cm}
\onslide<2->{\begin{block}{Beamer Introduction}
Beamer is a { \LaTeX} class.
\end{block}}
\vspace{1cm}
\onslide<3->{
Other choices: \tt{example}, \tt{lemma}, \tt{proof}.}
\end{frame}



%%%%%%%%%%%%%%%%%%%%%%%%%%%%%%Frame 17
\subsection{Lists}
\begin{frame}[fragile]
\frametitle{List-itemize}

\begin{columns}
\column{.5\textwidth}
\begin{block}
\scriptsize{
\begin{verbatim}
\begin{itemize}
\item The first one.
\item The second one.
\begin{itemize}
\item The larger one.
\item The smaller one.
\end{itemize}
\item The third one.
\end{itemize}
\end{verbatim}
}
\end{block}
\column{.5\textwidth}
\onslide<2>{
\begin{itemize}
\item The first one.
\item The second one.
\begin{itemize}
\item The larger one.
\item The smaller one.
\end{itemize}
\item The third one.
\end{itemize}}
\end{columns}

\end{frame}



%%%%%%%%%%%%%%%%%%%%%%%%%%%%%%Frame 18
\begin{frame}[fragile]
\frametitle{List-enumerate}

\begin{columns}
\column{.5\textwidth}
\begin{block}
\scriptsize{
\begin{verbatim}
\begin{enumerate}
\item The first one.
\item The second one.
\begin{enumerate}
\item The large one.
\item The small one.
\end{enumerate}
\item The third one.
\end{enumerate}
\end{verbatim}
}
\end{block}
\column{.5\textwidth}
\onslide<2>{
\begin{enumerate}
\item The first one.
\item The second one.
\begin{enumerate}
\item The large one.
\item The small one.
\end{enumerate}
\item The third one.
\end{enumerate}
}
\end{columns}


\end{frame}


%%%%%%%%%%%%%%%%%%%%%%%%%%%%%%Frame 19
\section{Overlays}


%%%%%%%%%%%%%%%%%%%%%%%%%%%%%%Frame 20
\begin{frame}[fragile]
\frametitle{Simple Overlays Using {\tt pause}}
\setbeamercovered{dynamic}
\begin{itemize}
\item  Beamer overlay frame 1\\ \pause
\item  Beamer overlay frame 2\\ \pause
\item  Beamer overlay frame 3\\ 
\end{itemize}
\pause
\scriptsize{
\begin{verbatim}
\begin{frame}[fragile]
\frametitle{Simple Overlays Using {\tt pause}}
\setbeamercovered{dynamic}
\begin{itemize}
\item  Beamer overlay frame 1\\ \pause
\item  Beamer overlay frame 2\\ \pause
\item  Beamer overlay frame 3\\ 
\end{itemize}
\end{verbatim}}
\end{frame}


%%%%%%%%%%%%%%%%%%%%%%%%%%%%%%Frame 21
\begin{frame}[fragile]
\frametitle{Advanced Overlays Using \tt{onslide}}
\setbeamercovered{invisible}
{\Huge
\begin{center}
\begin{tabular}{c|c|c}
\onslide<9->{8} & \onslide<8->{7} & \onslide<2->{1} \\ \hline
\onslide<6->{5} & \onslide<3->{2} & \onslide<5->{4} \\ \hline
\onslide<10->{9} & \onslide<7->{6} & \onslide<4->{3}
\end{tabular}
\end{center}
}
\end{frame}


%%%%%%%%%%%%%%%%%%%%%%%%%%%%%%Frame 22
\begin{frame}[fragile]
\frametitle{}
\begin{center}
\Huge{\color{blue}{Questions?}}
\end{center}
\end{frame}

\end{document}