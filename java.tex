\addcontentsline{to}{chapter}{}
\chapter*{JAVA}
\stepcounter{chapter}
Fue desarrollado por el equipo liderado por James Gosling en Sun Microsystems. Sun microsystems fue comprado por Oracle en 2010. Originalmente llamado Oak, Java fue diseñado en 1991 para uso embebido de chips en electrodomésticos de consumo. En 1995, renombrado Java, fue re diseñado para desarrollo de aplicaciones Web. 

Java se volvió enormemente popular. Fue descrito por sus diseñadores como \textit{ simple, orientado a objetos, distribuido, interpretado, robusto, seguro, neutral arquitecture, portable, de alto rendimiento, multiproceso y dinámico}.

Ahora Java es muy popular para desarrollo de aplicaciones en Servidores Web. Éstas aplicaciones procesan datos, realizan cálculos, y generan páginas web dinámicas. Muchos sitios web comerciales son desarrollados usando java en el backend. 

Java bien es 3 ediciones: \textit{Standard Edition, java Enterprise Edition (Java EE), y Java Micro Edition (Java ME)}. Además éstos están en diferentes ediciones.

\textit{ Java EE} Es para desarrollar aplicaciones en el lado del servidor, tales como Java servlets, páginas JavaServer (JSP), y JavaServer Faces (JSF).

\section{JDK}
Consiste de un conjunto de programas separados. El JDK está en su versión 19 en el momento actual pero LTS es la versión 17 \cite{historia}. 

\section{Versión de java que se utiliza en el proyecto}
\begin{verbatim}
tomas@debian:~$ java --version
openjdk 17.0.4 2022-07-19
OpenJDK Runtime Environment (build 17.0.4+8-Debian-1deb11u1)
OpenJDK 64-Bit Server VM (build 17.0.4+8-Debian-1deb11u1, mixed mode, sharing)
\end{verbatim}







