\addcontentsline{toc}{chapter}{Diseño de base de datos}
\chapter*{Diseño de base de datos}
\stepcounter{chapter}
\section{Diagrama entidad relación}
Se abstrae del problema real. Nuestras entidades serán el autor, 
Para el diseño entidad relación de base datos se utiliza la herramienta \textit{Draw io}. 

\begin{enumerate}
\item \textbf{DOI,} es un identificador único que permite identificar un artículo durante su vida útil aunque la revista en la que fue publicado desaparezca o cambia de nombre. Está compuesto por una parte que identifica a la entidad que registra el DOI y otra parte que identifica al objeto, en este caso, el artículo científico.
\end{enumerate}

%\section{MODELO VISTA CONTROLADOR}
%\subsection{MODELO}
%Mapeamos una clase con cada tabla, aquí unimos el código con la base de datos. 
%\subsection{VISTA}
%Aquí está la interfaz
%\subsection{CONTROLADOR}
%%También ya hay el paradigma de microservicios para construir el software
%\section{TECNOLOGÍA ORM UTILIZADO EN EL PROYECTO}
%\subsection{SPRING BOOT}
%%para node se usa express, para php láravel
